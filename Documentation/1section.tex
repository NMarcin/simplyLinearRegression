Tematem projektu jest pokazanie na przykładzie cen mieszkań w jaki sposób, używając zbioru danych, możemy przewidywać, jakie wartości będą miały kolejne pary danych. Technika używana przez nas do rozwiązania tego problemu to uzyskanie regresji liniowej przy użyciu spadku gradientowego. Cały problem składa się z dwóch części - teoretycznej oraz praktycznej.
	\subsection{Część teoretyczna}
W części teoretycznej zostaną przedstawione wszystkie obliczenia, które są niezbędne do wyznaczenia prostej, która w najdokładniejszy sposób opisze zbiór danych. Dzięki tej prostej będziemy w stanie ocenić najbardziej prawdopodobną daną wyjściową dla pewnym danych wejściowych. Do uzyskania tego policzymy funkcję kosztu dla kilku możliwych prostych, które będą w pewien sposób odwzorowywały te dane. Dzięki temu uzyskamy parabolę, dla której obliczymy globalne minimum za pomocą dwóch sposobów. Pierwszy używa wzór na funkcję tej paraboli, a drugi zaczyna od wybranego punkty i zbliża się do globalnego minimum za pomocą spadku gradientowego. 
	\subsection{Część praktyczna}
Praktyczna część będzie polegała na stworzenie programu, który sam będzie obliczał funkcję kosztu oraz stworzy z tego wykres kosztu, na którym potem przeprowadzi spadek gradientowy. Dzięki temu zabiegowi otrzymamy dokładnie taki sam wynik jak w części praktycznej. Całe oprogramowanie zostanie napisane w Octave - darmowym odpowiednikiem Matlab'a. To środowisko programistyczne zostało wybrane ze względu na prosty, a przede wszystkim czytelny sposób przeprowadzania obliczeń. Każda część będzie odwzorowana za pomocą wykresów, które będą również przedstawiały kroki, które były wykonywane w części teoretycznej.