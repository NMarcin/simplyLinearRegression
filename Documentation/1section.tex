Tematem projektu jest pokazanie na przykładzie cen mieszkań w jaki sposób, używając zbioru danych, możemy przewidywać, jakie wartości będą miały kolejne pary danych. Technika używana przez nas do rozwiązania tego problemu to uzyskanie regresji liniowej przy użyciu spadku gradientowego. Cały problem składa się z dwóch części - teoretycznej oraz praktycznej.
	\subsection{Część teoretyczna}
W części teoretycznej zostaną wytłumaczone wszystkie kroki na przykładowych obliczeniach, które są niezbędne do wyznaczenia prostej, która w najdokładniejszy sposób opisze zbiór danych. Dzięki tej prostej będziemy w stanie ocenić najbardziej prawdopodobną daną wyjściową dla pewnym danych wejściowych. Do uzyskania tego policzymy funkcję kosztu dla dwóch możliwych prostych, które będą w pewien sposób odwzorowywały te dane. Dzieki temu zrozumiemy na czym polega funkcja kosztu i dlaczego jej pochodna jest nam potrzebna do spadku gradientowego, który odnajdzie najlepsze $\theta_{0}$ oraz $\theta_{1}$.
	\subsection{Część praktyczna}
Praktyczna część będzie polegała na stworzenie programu, który sam będzie obliczał funkcję kosztu oraz stworzy z tego wykres kosztu, na którym potem przeprowadzi spadek gradientowy. Dzięki temu zabiegowi otrzymamy predykcję cen mieszkań w zależności od metrażu. Całe oprogramowanie zostanie napisane w Octave - darmowym odpowiednikiem Matlab'a. To środowisko programistyczne zostało wybrane ze względu na prosty, a przede wszystkim czytelny sposób przeprowadzania obliczeń. Każda część będzie odwzorowana za pomocą wykresów, które będą również przedstawiały kroki, tak aby jak najlepiej zoobrazować i wytłumaczyć w połączeniu z częścią teoretyczną ideę tego problemu. Spadek gradientowy dla tak błachego problemu jest przerostem formy nad treścią, aczkolwiek chodzi tutaj o jak najlepsze wytłumaczenie techniki.