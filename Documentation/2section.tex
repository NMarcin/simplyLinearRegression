	\subsection{Cel algorytmu}
Celem algorytmu jest przedstawienie regresji liniowej i jednego ze sposób, związengo z popularnym aktualnie uczeniem maszynowym, radzenia sobe z obliczeniem jej - spadku gradientowego. Przedstawiony tutaj algorytm będzie przewidywał cenę mieszkań we Wrocławiu w zależności od ilości metrów kwadratowych danego lokalu. W rzeczywistości na cenę mieszkania wpływa dużo więcej czynników, takich jak między innymi odległość od centrum, ilość przystanków, sklepów w okolicy czy możliwość zaparkowania pojazdu. Omawiany przez nas problem skupia się tylko i wyłącznie na wielkości mieszkania, ponieważ uznalismy to jako jeden z głównych czynników, z których wynika cena mieszkania. Oczywiście spadek gradientowy pozwala na dodanie dużo większej ilości danych wejściowych, wręcz nieskończonej (może to spowodować przeuczenie się algorytmu, co będzie powodowało dobre odwzorowanie tylko i wyłącznie dla zbioru testowego, a nie rzeczywistych danych). Zdecydowaliśmy jednak tak, ponieważ dzięki braniu pod uwagę tylko jednej danej wejścowej jesteśmy w stanie przedstawić każdy moment algorytmu używając wykresów w przestrzeni trójwymiarowej co naszym zdaniem jest dużo lepsze dla czytelnika, który pierwszy raz spotyka się z tym algorytmem.
	\subsection{Ograniczenia algorytmu}
Oczywiście jak większość algorytmów i ten posiada ograniczenia. Jak w większości algorytmów związanych z uczeniem maszynowym i sztuczną inteligencją bardzo łatwe jest wystąpienie niedouczenia lub przeuczenia. Pierwszy z problemów wystąpi na pewno - tak jak wspominaliśmy w poprzednim podpunkcie specjalnie nasz algorytm nie będzie idealnie odwzorowywał w pełni rzeczywistości ze względu na małą ilość parametrów wejściowych. Jednakże niedouczenie może wystąpić też z kilku innych powodów - jednym z głównych problemów algorytmów uczenia maszynowego jest zbiór danych, zazwyczaj posiadanie wystarczająco dużego i reprezentatywnego zbioru danych to jest 90\% sukcesu. 10\% jest to tylko i wyłącznie wybranie odpowiedniej metody oraz odpowiednie zmodyfikowanie jej pod dany problem. Kolejnym problemem jest przeuczenie naszeo algorytmu - z tym można poradzić sobie dużo łatwiej. Jedną z technik jest podzielenie zbioru danych na 3 części - 60\% danych są przeznaczane do nauczenia naszego algorytmu, 20\% do uzyskania najlepszej regresji liniowej oraz 20\% do ostatecznego przetestowania wynikowej. Przy przeuczeniu algorytmu możemy też spróbować zmniejszyć ilość parametrów wejściowych lub zmniejszych wagę części z nich używając tak zwanej regulacji spadku gradientowego. 