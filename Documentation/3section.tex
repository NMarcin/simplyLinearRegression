	\subsection{Pochodzenie danych}
Dane używane przez nas do algorytmu to para liczb - cena mieszkania oraz jego ilość metrów kwadratowych. Wszystkie dane zostały spisane z ogólnodostępnej strony otodom, gdzie każdy może wystawić mieszkanie na sprzedaż. Nie było żadnych innych kryteriów od tego, aby mieszkanie znajdowało się we Wrocławiu - są tu jednocześnie mieszkania używane jak i nowe budownictwo. Link do srony, z której zostały pobrane dane:
\\
\\
\url{https://www.otodom.pl/sprzedaz/mieszkanie/wroclaw/}
\\
\\
Tutaj dalszy tekst... Głównie pisane o tym ile danych, że brane losowo, niemodyfikowane itd.
	\subsection{Zoobrazowanie zbioru danych}
Pokazanie danych, wykresów z nimi itd
	\subsection{Filtracja}
Wywalenie 3 danych, czemu, żeby graniczne wywalić, bo psują zazwyczaj itd.	
	
	%\begin{figure}[H]
    %\centering
    %\includegraphics[scale=0.7]{zad2wykresy/ksi=1.png}
    %\caption{$\xi = 1$}
    %\label{lamana}
	%\end{figure}
	%
	%Otrzymaliśmy klasyczny człon inercyjny II rzędu.

	%\begin{figure}[H]
    %\centering
    %\includegraphics[scale=0.7]{zad2wykresy/ksi=-1.png}
    %\caption{$\xi = -1$}
    %\label{lamana}
	%\end{figure}
