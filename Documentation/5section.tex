\subsection{Wyznaczyłem za pomocą funkcji rfind K krytyczne, dla którego otrzymałem system na granicy stabilności}

	\begin{figure}[H]
    \centering
    \includegraphics[scale=0.7]{zad4wykresy/gain.PNG}
    \label{lamana}
	\end{figure}

 \begin{figure}[H]
    \centering
    \includegraphics[scale=0.7]{zad4wykresy/oscy.PNG}
    \label{lamana}
	\end{figure}
	
\subsection{Odczyt okresu oscylacji($P_u$) i wyznaczenie na jego podstawie $K_p$, $K_i$ oraz $K_d$ }
	
 \begin{tabular}{|l|c|r|r|}
  \hline Rodzaj regulacji & $K_p$ & $K_d$ & $K_i$\\
  \hline PID & 0.60 $K_u$ & $\frac{K_pP_u}{8}$ & $2.0 \frac{K_p}{P_u}$\\
  \hline PI & 0.45 $K_u$ & - & $1.2 \frac{K_p}{P_u}$\\
  \hline P & 0.50 $K_u$ & - & -\\
  \hline
 \end{tabular}
 \vspace{0.5cm}
 \newline
 Zmierzona wartość:
 \vspace{0.5cm}
 \newline
 \vspace{0.5cm}
 \LARGE
	 $K_u = 960  P_u = 1.6050s$
 \normalsize

\subsection{Prezentacja System z regulatorem PID}

	\begin{figure}[H]
    \centering
    \includegraphics[scale=0.7]{zad4wykresy/zad3.png}
    \label{lamana}
	\end{figure}
	
	\LARGE
   $Sys=\frac{1}{(s+3)(s+3)(s+5)(s+5)}$ \\
   $PID=K_p+\frac{K_i}{s} + K_d s$	\\
   $PID+Sys=\frac{PID*Sys}{PID*Sys+1}$
  \normalsize
	
\subsection{Dostosowywanie nastaw regulatora w zależności od oczekiwanej odpowiedzi skokowej}

	\begin{figure}[H]
    \centering
    \includegraphics[scale=0.7]{zad4wykresy/block_2.PNG}
    \label{lamana}
	\end{figure}
	
	\begin{figure}[H]
    \centering
    \includegraphics[scale=0.5]{zad4wykresy/tune.PNG}
    \label{lamana}
	\end{figure}
	

Narzędzie Tune w regulatorze PID służy do dobierania nastawy tak, aby  odpowiadał naszym wymaganiom, często używa się go do tego, by odpowiedź skokowa nie miała przeregulowań.